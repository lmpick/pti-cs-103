%% Generated by Sphinx.
\def\sphinxdocclass{report}
\documentclass[letterpaper,10pt,english,openany,oneside]{sphinxmanual}
\ifdefined\pdfpxdimen
   \let\sphinxpxdimen\pdfpxdimen\else\newdimen\sphinxpxdimen
\fi \sphinxpxdimen=.75bp\relax

\PassOptionsToPackage{warn}{textcomp}
\usepackage[utf8]{inputenc}
\ifdefined\DeclareUnicodeCharacter
% support both utf8 and utf8x syntaxes
\edef\sphinxdqmaybe{\ifdefined\DeclareUnicodeCharacterAsOptional\string"\fi}
  \DeclareUnicodeCharacter{\sphinxdqmaybe00A0}{\nobreakspace}
  \DeclareUnicodeCharacter{\sphinxdqmaybe2500}{\sphinxunichar{2500}}
  \DeclareUnicodeCharacter{\sphinxdqmaybe2502}{\sphinxunichar{2502}}
  \DeclareUnicodeCharacter{\sphinxdqmaybe2514}{\sphinxunichar{2514}}
  \DeclareUnicodeCharacter{\sphinxdqmaybe251C}{\sphinxunichar{251C}}
  \DeclareUnicodeCharacter{\sphinxdqmaybe2572}{\textbackslash}
\fi
\usepackage{cmap}
\usepackage[T1]{fontenc}
\usepackage{amsmath,amssymb,amstext}
\usepackage{babel}
\usepackage{times}
\usepackage[Bjarne]{fncychap}
\usepackage{sphinx}

\fvset{fontsize=\small}
\usepackage{geometry}

% Include hyperref last.
\usepackage{hyperref}
% Fix anchor placement for figures with captions.
\usepackage{hypcap}% it must be loaded after hyperref.
% Set up styles of URL: it should be placed after hyperref.
\urlstyle{same}

\addto\captionsenglish{\renewcommand{\figurename}{Fig.}}
\addto\captionsenglish{\renewcommand{\tablename}{Table}}
\addto\captionsenglish{\renewcommand{\literalblockname}{Listing}}

\addto\captionsenglish{\renewcommand{\literalblockcontinuedname}{continued from previous page}}
\addto\captionsenglish{\renewcommand{\literalblockcontinuesname}{continues on next page}}
\addto\captionsenglish{\renewcommand{\sphinxnonalphabeticalgroupname}{Non-alphabetical}}
\addto\captionsenglish{\renewcommand{\sphinxsymbolsname}{Symbols}}
\addto\captionsenglish{\renewcommand{\sphinxnumbersname}{Numbers}}

\addto\extrasenglish{\def\pageautorefname{page}}

\setcounter{tocdepth}{0}



\title{pti\_cs\_lectures Documentation}
\date{Feb 02, 2019}
\release{}
\author{PTI}
\newcommand{\sphinxlogo}{\vbox{}}
\renewcommand{\releasename}{}
\makeindex
\begin{document}

\pagestyle{empty}
\sphinxmaketitle
\pagestyle{plain}
\sphinxtableofcontents
\pagestyle{normal}
\phantomsection\label{\detokenize{index::doc}}



\chapter{Introduction}
\label{\detokenize{index:introduction}}

\chapter{Syllabus}
\label{\detokenize{index:syllabus}}

\section{Introduction}
\label{\detokenize{introduction:introduction}}\label{\detokenize{introduction::doc}}
\begin{sphinxadmonition}{note}{\label{introduction:index-0}Todo:}\begin{itemize}
\item {} 
Put together all references at the end

\item {} 
Ask Sedgewick for permission to copy HelloWorld

\item {} 
Put in references to lecture names for hardware and software instead of lecture 3 and lecture 4 (respectively)

\item {} 
Add pictures

\item {} 
Add course expectations

\item {} 
Add more discussion questions

\item {} 
Add section for the algorithmic thinking activity

\end{itemize}
\end{sphinxadmonition}


\subsection{Overview of course}
\label{\detokenize{introduction:overview-of-course}}
Knowing just a little bit of computer science can get you started right away in actual applications. One of the goals of this course is to learn about the fascinating subject of computer science. Another is to develop algorithmic thinking skills that will help with day-to-day critical problem-solving skills. But perhaps the most important goal of the course is to develop coding skills, which will not only open up new job opportunities but also make you more effective in most areas of business.

In the first semester, we will spend the first two classes of each week on computer science theory and special topics. The final day of each week will be a lab day, where we actually start practicing coding skills.

In the second semester, we will start focusing more on practical coding, with a single day a week for theory and 2 lab periods per week for coding.

Broadly, we will cover the following topics:
\begin{itemize}
\item {} \begin{description}
\item[{How modern computers work}] \leavevmode\begin{itemize}
\item {} 
Hardware

\item {} 
Software

\item {} 
Computer networks and information systems

\end{itemize}

\end{description}

\item {} \begin{description}
\item[{Algorithms for quickly solving complex problems}] \leavevmode\begin{itemize}
\item {} 
Searching

\item {} 
Sorting

\end{itemize}

\end{description}

\item {} \begin{description}
\item[{Data structures}] \leavevmode\begin{itemize}
\item {} 
Arrays

\item {} 
ArrayLists

\end{itemize}

\end{description}

\item {} \begin{description}
\item[{Applications of Computer science}] \leavevmode\begin{itemize}
\item {} 
Basic coding in Java

\item {} 
How to use productivity software

\end{itemize}

\end{description}

\end{itemize}


\subsection{Brief history of computer science}
\label{\detokenize{introduction:brief-history-of-computer-science}}
\sphinxhref{https://www.worldsciencefestival.com/infographics/a\_history\_of\_computer\_science/}{Timeline}:
\begin{itemize}
\item {} 
Invention of the abacus (2700-2300 BC, Sumerians)

\item {} 
Design of first modern-style computer (Charles Babbage, 1837)

\item {} 
Design of first computer algorithm (Ada Lovelace, 1843)

\item {} 
Invention of first electronic digital computer (Konrad Zuse, 1941)

\item {} 
Invention of the transistor (Bell labs, 1947)

\item {} 
Invention of the first computer network (early Internet) (DARPA, 1968)

\item {} 
Invention of the World Wide Web (Sir Tim Berners-Lee, 1990)

\end{itemize}


\subsection{Components of a computer}
\label{\detokenize{introduction:components-of-a-computer}}
A computer is an electronic device used to process data. Its basic role is to convert data into information that is useful to people.

There are 4 primary components of a computer:
\begin{itemize}
\item {} 
Hardware

\item {} 
Software

\item {} 
Data

\item {} 
User

\end{itemize}


\subsubsection{Hardware}
\label{\detokenize{introduction:hardware}}
Computer hardware consists of physical, electronic devices. These are the parts you actually can see and touch. Some examples include
\begin{itemize}
\item {} 
Central processing unit (CPU)

\item {} 
Monitor

\item {} 
CD drive

\item {} 
Keyboard

\item {} 
Computer data storage

\item {} 
Graphic card

\item {} 
Sound card

\item {} 
Speakers

\item {} 
Motherboard

\end{itemize}

We will discuss these components in more detail in lecture 3.


\subsubsection{Software}
\label{\detokenize{introduction:software}}
Software (otherwise known as “programs” or “applications”) are organized sets of instructions for controlling the computer.

There are two main classes of software:
\begin{itemize}
\item {} 
Applications software: programs allowing the human to interact directly with the computer

\item {} 
Systems software: programs the computer uses to control itself

\end{itemize}

Some more familiar applications software include
\begin{itemize}
\item {} 
Microsoft Word: allows the user to edit text files

\item {} 
Internet Explorer: connects the user to the world wide web

\item {} 
iTunes: organizes and plays music files

\end{itemize}

While systems software allows the user to interact with the computer, systems software keeps the computer running. The operating system (OS) is the most common example of systems software, and it schedules tasks and manages storage of data.

We will dive deeper into the details of both applications and systems software in lecture 4.


\subsubsection{Data}
\label{\detokenize{introduction:data}}
Data is fundamentally information of any kind. One key benefit of computers is their ability to reliably store massive quantities of data for a long time. Another is the speed with which they can do calculations on data once they recieve instructions from a human user.

While humans can understand data with a wide variety of perceptions (taste, smell, hearing, touch, sight), computers read and write everything internally as “bits”, or 0s and 1s.

Computers have software and hardware which allow them to convert their internal 0s and 1s into text, numerals, and images displayed on the monitor; and sounds which can be played through the speaker.

Similarly, humans have hardware and software used for converting human signals into computer-readable signals: a microphone converts sound, a camera converts pictures, and a text editor converts character symbols.


\subsubsection{Users}
\label{\detokenize{introduction:users}}
Of course, there would be no data and no meaningful calculations without the human user. Computers are ultimately tools for making humans more powerful.

As we will see in the next section, however, different types of computers have different roles for the user.


\subsection{Types of computers}
\label{\detokenize{introduction:types-of-computers}}

\subsubsection{Supercomputers}
\label{\detokenize{introduction:supercomputers}}
These are the most powerful computers out there. The are used for problems that take a long time to calculate. They are rare because of their size and expense, and therefore primarily used by big organizations like universities, corporations, and the government.

The user of a supercomputer typically gives the computer a list of instructions, and allows the supercomputer to run on its own over the course of hours or days to complete its task.


\subsubsection{Mainframe computers}
\label{\detokenize{introduction:mainframe-computers}}
Although not as powerful as supercomputers, mainframe computers can handle more data and run much faster than a typical personal computer. Often, they are given instructions only periodically by computer programmers, and then run on their own for months at a time to store and process incoming data. For example, census number-crunching, consumer statistics, and transactions processing all use mainframe computers.


\subsubsection{Personal computers}
\label{\detokenize{introduction:personal-computers}}
These are the familiar computers we use to interact with applications every day. Full-size desktop computers and laptop computers are examples.


\subsubsection{Embedded computers}
\label{\detokenize{introduction:embedded-computers}}
In the modern “digital” age, nearly all devices we use have computers embedded within them. From cars to washing machines to watches to heating systems, most everyday appliances have a computer within them that allows them to function.


\subsubsection{Mobile computers}
\label{\detokenize{introduction:mobile-computers}}
In the past 2 decades, mobile devices have exploded onto the scene, and smartphones have essentially become as capable as standalone personal computers for many tasks.


\subsection{Why computers are useful}
\label{\detokenize{introduction:why-computers-are-useful}}
Computers help us in most tasks in the modern age. We can use them, for example, to
\begin{itemize}
\item {} 
write a letter

\item {} 
do our taxes

\item {} 
play video games

\item {} 
watch videos

\item {} 
surf the internet

\item {} 
keep in touch with friends

\item {} 
date

\item {} 
order food

\item {} 
control robots and self-driving cars

\end{itemize}

\sphinxstylestrong{Question:} What are some other tasks a computer can accomplish?

This is why the job market for computer scientists continues to expand, and why computer skills are more and more necessary even in non-computational jobs.

According to a \sphinxhref{https://insights.stackoverflow.com/survey/2018/}{Stackoverflow survey from 2018}, 11\% of coders have only been at it for 0-2 years. This demonstrates two things:
\begin{enumerate}
\def\theenumi{\arabic{enumi}}
\def\labelenumi{\theenumi .}
\makeatletter\def\p@enumii{\p@enumi \theenumi .}\makeatother
\item {} 
The job market for people with coding skills is continually expanding

\item {} 
It doesn’t take much to become a coder

\end{enumerate}

Some examples of careers in computer science include
\begin{itemize}
\item {} 
IT management / consulting

\item {} 
Game developer

\item {} 
Web developer

\item {} 
UI/UX designer

\item {} 
Data analyst

\item {} 
Database manager

\end{itemize}


\subsection{First program}
\label{\detokenize{introduction:first-program}}
This entire section is taken from the \sphinxhref{https://introcs.cs.princeton.edu/java/11hello/}{intro to CS course at Princeton}.

We will now write our first program in Java, which will demonstrate the 3 basic steps to get a simple program running.
\begin{enumerate}
\def\theenumi{\arabic{enumi}}
\def\labelenumi{\theenumi .}
\makeatletter\def\p@enumii{\p@enumi \theenumi .}\makeatother
\item {} 
Create the program by typing it into a text editor and saving it to, for example, \sphinxtitleref{MyProgram.java}

\item {} 
Compile it by typing \sphinxtitleref{javac MyProgram.java} in the terminal window.

\item {} 
Execute (or run) it by typing \sphinxtitleref{java MyProgram} in the terminal window.

\end{enumerate}

The first step creates the program; the second translates it into a language more suitable for machine execution (and puts the result in a file named MyProgram.class); the third actually runs the program.


\subsubsection{Creating a Java program}
\label{\detokenize{introduction:creating-a-java-program}}
A program is nothing more than a sequence of characters, like a sentence, a paragraph, or a poem. To create one, we need only define that sequence characters using a text editor in the same way as we do for email. \sphinxtitleref{HelloWorld.java} is an example program. Type these character into your text editor and save it into a file named \sphinxtitleref{HelloWorld.java}.

\begin{sphinxVerbatim}[commandchars=\\\{\}]
\PYG{k+kd}{public} \PYG{k+kd}{class} \PYG{n+nc}{HelloWorld} \PYG{o}{\PYGZob{}}
   \PYG{k+kd}{public} \PYG{k+kd}{static} \PYG{k+kt}{void} \PYG{n+nf}{main}\PYG{o}{(}\PYG{n}{String}\PYG{o}{[}\PYG{o}{]} \PYG{n}{args}\PYG{o}{)} \PYG{o}{\PYGZob{}}
      \PYG{c+c1}{// Prints \PYGZdq{}Hello, World\PYGZdq{} in the terminal window.}
      \PYG{n}{System}\PYG{o}{.}\PYG{n+na}{out}\PYG{o}{.}\PYG{n+na}{println}\PYG{o}{(}\PYG{l+s}{\PYGZdq{}Hello, World\PYGZdq{}}\PYG{o}{)}\PYG{o}{;}
   \PYG{o}{\PYGZcb{}}
\PYG{o}{\PYGZcb{}}
\end{sphinxVerbatim}


\subsubsection{Compiling a Java program}
\label{\detokenize{introduction:compiling-a-java-program}}
A compiler is an application that translates programs from the Java language to a language more suitable for executing on the computer. It takes a text file with the \sphinxtitleref{.java} extension as input (your program) and produces a file with a \sphinxtitleref{.class} extension (the computer-language version). To compile \sphinxtitleref{HelloWorld.java} type the text below at the terminal. (We use the \% symbol to denote the command prompt, but it may appear different depending on your system.):

\begin{sphinxVerbatim}[commandchars=\\\{\}]
\PYG{o}{\PYGZpc{}} \PYG{n}{javac} \PYG{n}{HelloWorld}\PYG{o}{.}\PYG{n}{java}
\end{sphinxVerbatim}

If you typed in the program correctly, you should see no error messages. Otherwise, go back and make sure you typed in the program exactly as it appears above.


\subsubsection{Executing (or running) a Java program}
\label{\detokenize{introduction:executing-or-running-a-java-program}}
Once you compile your program, you can execute it. This is the exciting part, where the computer follows your instructions. To run the HelloWorld program, type the following in the terminal window::

\begin{sphinxVerbatim}[commandchars=\\\{\}]
\PYG{o}{\PYGZpc{}} \PYG{n}{java} \PYG{n}{HelloWorld}
\end{sphinxVerbatim}

If all goes well, you should see the following response:

\begin{sphinxVerbatim}[commandchars=\\\{\}]
\PYG{n}{Hello}\PYG{p}{,} \PYG{n}{world}
\end{sphinxVerbatim}


\subsubsection{Understanding a Java program}
\label{\detokenize{introduction:understanding-a-java-program}}
The key line with \sphinxtitleref{System.out.println()} prints the text “Hello, World” in the terminal window. When we begin to write more complicated programs, we will discuss the meaning of \sphinxtitleref{public}, \sphinxtitleref{class}, \sphinxtitleref{main}, \sphinxtitleref{String{[}{]}}, \sphinxtitleref{args}, \sphinxtitleref{System.out}, and so on.


\subsubsection{Creating your own Java program}
\label{\detokenize{introduction:creating-your-own-java-program}}
For the time being, all of our programs will be just like \sphinxtitleref{HelloWorld.java}, except with a different sequence of statements in \sphinxtitleref{main()}. The easiest way to write such a program is to:
\begin{itemize}
\item {} 
Copy \sphinxtitleref{HelloWorld.java} into a new file whose name is the program name followed by \sphinxtitleref{.java}.

\item {} 
Replace \sphinxtitleref{HelloWorld} with the program name everywhere.

\item {} 
Replace the print statement by a sequence of statements.

\end{itemize}

\sphinxstylestrong{Exercise:}
Create your own program, \sphinxtitleref{HelloMe.java}, that prints out “Hello \sphinxtitleref{name}” with your own name in place of \sphinxtitleref{name}.


\section{Arrays}
\label{\detokenize{arrays-arraylists:arrays}}\label{\detokenize{arrays-arraylists::doc}}

\subsection{Topics}
\label{\detokenize{arrays-arraylists:topics}}\begin{enumerate}
\def\theenumi{\arabic{enumi}}
\def\labelenumi{\theenumi .}
\makeatletter\def\p@enumii{\p@enumi \theenumi .}\makeatother
\item {} 
Motivation

\item {} 
Creating arrays

\item {} 
Programming with arrays

\item {} 
Exchanging and shuffling

\item {} 
Exercises

\end{enumerate}


\subsection{Motivation}
\label{\detokenize{arrays-arraylists:motivation}}
Consider this snippet of code:

\begin{sphinxVerbatim}[commandchars=\\\{\}]
\PYG{k}{if}      \PYG{p}{(}\PYG{n}{day} \PYG{o}{==}  \PYG{l+m+mi}{0}\PYG{p}{)} \PYG{n}{System}\PYG{o}{.}\PYG{n}{out}\PYG{o}{.}\PYG{n}{println}\PYG{p}{(}\PYG{l+s+s2}{\PYGZdq{}}\PYG{l+s+s2}{Monday}\PYG{l+s+s2}{\PYGZdq{}}\PYG{p}{)}\PYG{p}{;}
\PYG{k}{else} \PYG{k}{if} \PYG{p}{(}\PYG{n}{day} \PYG{o}{==}  \PYG{l+m+mi}{1}\PYG{p}{)} \PYG{n}{System}\PYG{o}{.}\PYG{n}{out}\PYG{o}{.}\PYG{n}{println}\PYG{p}{(}\PYG{l+s+s2}{\PYGZdq{}}\PYG{l+s+s2}{Tuesday}\PYG{l+s+s2}{\PYGZdq{}}\PYG{p}{)}\PYG{p}{;}
\PYG{k}{else} \PYG{k}{if} \PYG{p}{(}\PYG{n}{day} \PYG{o}{==}  \PYG{l+m+mi}{2}\PYG{p}{)} \PYG{n}{System}\PYG{o}{.}\PYG{n}{out}\PYG{o}{.}\PYG{n}{println}\PYG{p}{(}\PYG{l+s+s2}{\PYGZdq{}}\PYG{l+s+s2}{Wednesday}\PYG{l+s+s2}{\PYGZdq{}}\PYG{p}{)}\PYG{p}{;}
\PYG{k}{else} \PYG{k}{if} \PYG{p}{(}\PYG{n}{day} \PYG{o}{==}  \PYG{l+m+mi}{3}\PYG{p}{)} \PYG{n}{System}\PYG{o}{.}\PYG{n}{out}\PYG{o}{.}\PYG{n}{println}\PYG{p}{(}\PYG{l+s+s2}{\PYGZdq{}}\PYG{l+s+s2}{Thursday}\PYG{l+s+s2}{\PYGZdq{}}\PYG{p}{)}\PYG{p}{;}
\PYG{k}{else} \PYG{k}{if} \PYG{p}{(}\PYG{n}{day} \PYG{o}{==}  \PYG{l+m+mi}{4}\PYG{p}{)} \PYG{n}{System}\PYG{o}{.}\PYG{n}{out}\PYG{o}{.}\PYG{n}{println}\PYG{p}{(}\PYG{l+s+s2}{\PYGZdq{}}\PYG{l+s+s2}{Friday}\PYG{l+s+s2}{\PYGZdq{}}\PYG{p}{)}\PYG{p}{;}
\PYG{k}{else} \PYG{k}{if} \PYG{p}{(}\PYG{n}{day} \PYG{o}{==}  \PYG{l+m+mi}{5}\PYG{p}{)} \PYG{n}{System}\PYG{o}{.}\PYG{n}{out}\PYG{o}{.}\PYG{n}{println}\PYG{p}{(}\PYG{l+s+s2}{\PYGZdq{}}\PYG{l+s+s2}{Saturday}\PYG{l+s+s2}{\PYGZdq{}}\PYG{p}{)}\PYG{p}{;}
\PYG{k}{else} \PYG{k}{if} \PYG{p}{(}\PYG{n}{day} \PYG{o}{==}  \PYG{l+m+mi}{6}\PYG{p}{)} \PYG{n}{System}\PYG{o}{.}\PYG{n}{out}\PYG{o}{.}\PYG{n}{println}\PYG{p}{(}\PYG{l+s+s2}{\PYGZdq{}}\PYG{l+s+s2}{Sunday}\PYG{l+s+s2}{\PYGZdq{}}\PYG{p}{)}\PYG{p}{;}
\end{sphinxVerbatim}
\begin{quote}

\sphinxstylestrong{Question:} What does this code do?
\end{quote}

This code prints the day of the week after conditioning on the value of an integer \sphinxtitleref{day}. But this code is repetitive. It would be useful if we had some way of creating a list of days of the week, and then just specifying which of those days we wanted to print. Something like this:

\begin{sphinxVerbatim}[commandchars=\\\{\}]
\PYG{n}{System}\PYG{o}{.}\PYG{n}{out}\PYG{o}{.}\PYG{n}{println}\PYG{p}{(}\PYG{n}{DAYS\PYGZus{}OF\PYGZus{}WEEK}\PYG{p}{[}\PYG{n}{day}\PYG{p}{]}\PYG{p}{)}\PYG{p}{;}
\end{sphinxVerbatim}

To achieve this in Java, we need arrays. An \sphinxstyleemphasis{array} is an ordered and fixed-length list of values that are of the same type. We can access data in an array by \sphinxstyleemphasis{indexing}, which means referring to specific values in the array by number. If an array has \sphinxcode{\sphinxupquote{n}} values, then we think of it as being numbered from \sphinxcode{\sphinxupquote{0}} to \sphinxcode{\sphinxupquote{n-1}}.


\subsection{Creating arrays}
\label{\detokenize{arrays-arraylists:creating-arrays}}
The syntax for creating an array in Java has three parts:
\begin{enumerate}
\def\theenumi{\arabic{enumi}}
\def\labelenumi{\theenumi .}
\makeatletter\def\p@enumii{\p@enumi \theenumi .}\makeatother
\item {} 
Array type

\item {} 
Array name

\item {} 
Either: array size or specific values

\end{enumerate}

For example, this code creates an array of size \sphinxcode{\sphinxupquote{n = 10}} and fills it with all :code:{\color{red}\bfseries{}{}`}0.0{}`s.

\begin{sphinxVerbatim}[commandchars=\\\{\}]
\PYG{n}{double}\PYG{p}{[}\PYG{p}{]} \PYG{n}{arr}\PYG{p}{;}                    \PYG{o}{/}\PYG{o}{/} \PYG{n}{Declare} \PYG{n}{array}
\PYG{n}{arr} \PYG{o}{=} \PYG{n}{new} \PYG{n}{double}\PYG{p}{[}\PYG{n}{n}\PYG{p}{]}\PYG{p}{;}             \PYG{o}{/}\PYG{o}{/} \PYG{n}{Initialize} \PYG{n}{the} \PYG{n}{array}
\PYG{k}{for} \PYG{p}{(}\PYG{n+nb}{int} \PYG{n}{i} \PYG{o}{=} \PYG{l+m+mi}{0}\PYG{p}{;} \PYG{n}{i} \PYG{o}{\PYGZlt{}} \PYG{n}{n}\PYG{p}{;} \PYG{n}{i}\PYG{o}{+}\PYG{o}{+}\PYG{p}{)} \PYG{p}{\PYGZob{}}    \PYG{o}{/}\PYG{o}{/} \PYG{n}{Iterate} \PYG{n}{over} \PYG{n}{array}
    \PYG{n}{arr}\PYG{p}{[}\PYG{n}{i}\PYG{p}{]} \PYG{o}{=} \PYG{l+m+mf}{0.0}\PYG{p}{;}                 \PYG{o}{/}\PYG{o}{/} \PYG{n}{Initialize} \PYG{n}{elements} \PYG{n}{to} \PYG{l+m+mf}{0.0}
\PYG{p}{\PYGZcb{}}
\end{sphinxVerbatim}

The key steps are: we first declare and initialize the array. We then loop over the array to initialize specific values. We can also initialize the array at compile time, for example

\begin{sphinxVerbatim}[commandchars=\\\{\}]
\PYG{n}{String}\PYG{p}{[}\PYG{p}{]} \PYG{n}{DAYS\PYGZus{}OF\PYGZus{}WEEK} \PYG{o}{=} \PYG{p}{\PYGZob{}}
\PYG{o}{/}\PYG{o}{/}  \PYG{n}{Indices}\PYG{p}{:}
\PYG{o}{/}\PYG{o}{/}  \PYG{l+m+mi}{0}         \PYG{l+m+mi}{1}          \PYG{l+m+mi}{2}            \PYG{l+m+mi}{3}           \PYG{l+m+mi}{4}         \PYG{l+m+mi}{5}           \PYG{l+m+mi}{6}
    \PYG{l+s+s2}{\PYGZdq{}}\PYG{l+s+s2}{Monday}\PYG{l+s+s2}{\PYGZdq{}}\PYG{p}{,} \PYG{l+s+s2}{\PYGZdq{}}\PYG{l+s+s2}{Tuesday}\PYG{l+s+s2}{\PYGZdq{}}\PYG{p}{,} \PYG{l+s+s2}{\PYGZdq{}}\PYG{l+s+s2}{Wednesday}\PYG{l+s+s2}{\PYGZdq{}}\PYG{p}{,} \PYG{l+s+s2}{\PYGZdq{}}\PYG{l+s+s2}{Thursday}\PYG{l+s+s2}{\PYGZdq{}}\PYG{p}{,} \PYG{l+s+s2}{\PYGZdq{}}\PYG{l+s+s2}{Friday}\PYG{l+s+s2}{\PYGZdq{}}\PYG{p}{,} \PYG{l+s+s2}{\PYGZdq{}}\PYG{l+s+s2}{Saturday}\PYG{l+s+s2}{\PYGZdq{}}\PYG{p}{,} \PYG{l+s+s2}{\PYGZdq{}}\PYG{l+s+s2}{Sunday}\PYG{l+s+s2}{\PYGZdq{}}
\PYG{p}{\PYGZcb{}}\PYG{p}{;}
\end{sphinxVerbatim}

Notice the difference in syntax. When creating an empty array, we must specify a size. When initialize an array at compile time with specific values, the size is implicit in the number of values provided.

Finally, in Java, it is acceptable to move the brackets to directly after the type declaration to directly after the name declaration. For example, these two declarations are equivalent:

\begin{sphinxVerbatim}[commandchars=\\\{\}]
\PYG{n+nb}{int} \PYG{n}{arr}\PYG{p}{[}\PYG{p}{]}\PYG{p}{;}
\PYG{n+nb}{int}\PYG{p}{[}\PYG{p}{]} \PYG{n}{arr}\PYG{p}{;}
\end{sphinxVerbatim}


\subsection{Programming with arrays}
\label{\detokenize{arrays-arraylists:programming-with-arrays}}

\subsubsection{Indexing}
\label{\detokenize{arrays-arraylists:indexing}}
Consider the array \sphinxcode{\sphinxupquote{DAYS\_OF\_WEEK}} from the previous section. We can \sphinxstyleemphasis{index} the array using the following syntax:

\begin{sphinxVerbatim}[commandchars=\\\{\}]
\PYG{n}{System}\PYG{o}{.}\PYG{n}{out}\PYG{o}{.}\PYG{n}{println}\PYG{p}{(}\PYG{n}{DAYS\PYGZus{}OF\PYGZus{}WEEK}\PYG{p}{[}\PYG{l+m+mi}{3}\PYG{p}{]}\PYG{p}{)}\PYG{p}{;}  \PYG{o}{/}\PYG{o}{/} \PYG{n}{Prints} \PYG{l+s+s2}{\PYGZdq{}}\PYG{l+s+s2}{Thursday}\PYG{l+s+s2}{\PYGZdq{}}
\end{sphinxVerbatim}

In Java, array’s are said to use \sphinxstyleemphasis{zero-based indexing} because the first element in the array is accessed with the number \sphinxcode{\sphinxupquote{0}} rather than \sphinxtitleref{1}.
\begin{quote}

\sphinxstylestrong{Question:} What does \sphinxcode{\sphinxupquote{System.out.println(DAYS\_OF\_WEEK{[}1{]});}} print?

\sphinxstylestrong{Question:} What does this code do? What number does it print?

\begin{sphinxVerbatim}[commandchars=\\\{\}]
\PYG{n}{double} \PYG{n+nb}{sum} \PYG{o}{=} \PYG{l+m+mf}{0.0}\PYG{p}{;}
\PYG{n}{double}\PYG{p}{[}\PYG{p}{]} \PYG{n}{arr} \PYG{o}{=} \PYG{p}{\PYGZob{}} \PYG{l+m+mi}{1}\PYG{p}{,} \PYG{l+m+mi}{2}\PYG{p}{,} \PYG{l+m+mi}{2}\PYG{p}{,} \PYG{l+m+mi}{3}\PYG{p}{,} \PYG{l+m+mi}{4}\PYG{p}{,} \PYG{l+m+mi}{7}\PYG{p}{,} \PYG{l+m+mi}{9} \PYG{p}{\PYGZcb{}}
\PYG{k}{for} \PYG{p}{(}\PYG{n+nb}{int} \PYG{n}{i} \PYG{o}{=} \PYG{l+m+mi}{0}\PYG{p}{;} \PYG{n}{i} \PYG{o}{\PYGZlt{}} \PYG{n}{arr}\PYG{o}{.}\PYG{n}{length}\PYG{p}{;} \PYG{n}{i}\PYG{o}{+}\PYG{o}{+}\PYG{p}{)} \PYG{p}{\PYGZob{}}
    \PYG{n+nb}{sum} \PYG{o}{+}\PYG{o}{=} \PYG{n}{arr}\PYG{p}{[}\PYG{n}{i}\PYG{p}{]}\PYG{p}{;}
\PYG{p}{\PYGZcb{}}
\PYG{n}{System}\PYG{o}{.}\PYG{n}{out}\PYG{o}{.}\PYG{n}{println}\PYG{p}{(}\PYG{n+nb}{sum} \PYG{o}{/} \PYG{n}{arr}\PYG{o}{.}\PYG{n}{length}\PYG{p}{)}\PYG{p}{;}
\end{sphinxVerbatim}
\end{quote}


\subsubsection{Array length}
\label{\detokenize{arrays-arraylists:array-length}}
As mentioned previously, arrays are \sphinxstyleemphasis{fixed-length}. After you have created an array, it’s length is unchangeable. You can access the length of an array \sphinxcode{\sphinxupquote{arr{[}{]}}} with the code \sphinxcode{\sphinxupquote{arr.length}}.
\begin{quote}

\sphinxstylestrong{Question:} What does \sphinxcode{\sphinxupquote{System.out.println(DAYS\_OF\_WEEK.length);}} print?

\sphinxstylestrong{Exercise:} Write a \sphinxcode{\sphinxupquote{for}} loop to print the days of the week in order (Monday through Sunday) using an array rather than seven \sphinxcode{\sphinxupquote{System.out.println}} function calls.
\end{quote}


\subsubsection{Default initialization}
\label{\detokenize{arrays-arraylists:default-initialization}}
In Java, the default initial values for numeric primitive types is \sphinxcode{\sphinxupquote{0}} and \sphinxcode{\sphinxupquote{false}} for the \sphinxcode{\sphinxupquote{boolean}} type.
\begin{quote}

\sphinxstylestrong{Exercise:} Consider this code from earlier:

\begin{sphinxVerbatim}[commandchars=\\\{\}]
\PYG{n}{double}\PYG{p}{[}\PYG{p}{]} \PYG{n}{arr}\PYG{p}{;}
\PYG{n}{arr} \PYG{o}{=} \PYG{n}{new} \PYG{n}{double}\PYG{p}{[}\PYG{n}{n}\PYG{p}{]}\PYG{p}{;}
\PYG{k}{for} \PYG{p}{(}\PYG{n+nb}{int} \PYG{n}{i} \PYG{o}{=} \PYG{l+m+mi}{0}\PYG{p}{;} \PYG{n}{i} \PYG{o}{\PYGZlt{}} \PYG{n}{n}\PYG{p}{;} \PYG{n}{i}\PYG{o}{+}\PYG{o}{+}\PYG{p}{)} \PYG{p}{\PYGZob{}}
    \PYG{n}{arr}\PYG{p}{[}\PYG{n}{i}\PYG{p}{]} \PYG{o}{=} \PYG{l+m+mf}{0.0}\PYG{p}{;}
\PYG{p}{\PYGZcb{}}

\PYG{n}{Rewrite} \PYG{n}{this} \PYG{n}{code} \PYG{n}{to} \PYG{n}{be} \PYG{n}{a} \PYG{n}{single} \PYG{n}{line}\PYG{o}{.}
\end{sphinxVerbatim}
\end{quote}


\subsubsection{Bounds checking}
\label{\detokenize{arrays-arraylists:bounds-checking}}
Consider this snippet of code.
\begin{quote}

\sphinxstylestrong{Question:} Where is the bug?

\begin{sphinxVerbatim}[commandchars=\\\{\}]
\PYG{n+nb}{int}\PYG{p}{[}\PYG{p}{]} \PYG{n}{arr} \PYG{o}{=} \PYG{n}{new} \PYG{n+nb}{int}\PYG{p}{[}\PYG{l+m+mi}{100}\PYG{p}{]}\PYG{p}{;}
\PYG{k}{for} \PYG{p}{(}\PYG{n+nb}{int} \PYG{n}{i} \PYG{o}{=} \PYG{l+m+mi}{0}\PYG{p}{;} \PYG{n}{i} \PYG{o}{\PYGZlt{}}\PYG{o}{=} \PYG{l+m+mi}{100}\PYG{p}{;} \PYG{o}{+}\PYG{o}{+}\PYG{n}{i}\PYG{p}{)} \PYG{p}{\PYGZob{}}
    \PYG{n}{System}\PYG{o}{.}\PYG{n}{out}\PYG{o}{.}\PYG{n}{println}\PYG{p}{(}\PYG{n}{arr}\PYG{p}{[}\PYG{n}{i}\PYG{p}{]}\PYG{p}{)}\PYG{p}{;}
\PYG{p}{\PYGZcb{}}
\end{sphinxVerbatim}
\end{quote}

The issue is that the program attempts to access the value \sphinxcode{\sphinxupquote{arr{[}100{]}}}, while the last element in the array is \sphinxcode{\sphinxupquote{arr{[}99{]}}}.

This kind of bug is called an “off-by-one error” and is so common… well, it has a name. In general, an off-by-one-error is one in which a loop iterates one time too many or too few.
\begin{quote}

\sphinxstylestrong{Question:} Where is the off-by-one-error?

\begin{sphinxVerbatim}[commandchars=\\\{\}]
\PYG{n+nb}{int}\PYG{p}{[}\PYG{p}{]} \PYG{n}{arr} \PYG{o}{=} \PYG{n}{new} \PYG{n+nb}{int}\PYG{p}{[}\PYG{l+m+mi}{100}\PYG{p}{]}\PYG{p}{;}
\PYG{k}{for} \PYG{p}{(}\PYG{n+nb}{int} \PYG{n}{i} \PYG{o}{=} \PYG{l+m+mi}{100}\PYG{p}{;} \PYG{n}{i} \PYG{o}{\PYGZgt{}} \PYG{l+m+mi}{0}\PYG{p}{;} \PYG{o}{\PYGZhy{}}\PYG{o}{\PYGZhy{}}\PYG{n}{i}\PYG{p}{)} \PYG{p}{\PYGZob{}}
    \PYG{n}{System}\PYG{o}{.}\PYG{n}{out}\PYG{o}{.}\PYG{n}{println}\PYG{p}{(}\PYG{n}{arr}\PYG{p}{[}\PYG{n}{i}\PYG{p}{]}\PYG{p}{)}\PYG{p}{;}
\PYG{p}{\PYGZcb{}}
\end{sphinxVerbatim}

\sphinxstylestrong{Exercise:} Fill in the missing code in this \sphinxcode{\sphinxupquote{for}} loop to print the numbers in reverse order, i.e. \sphinxcode{\sphinxupquote{5, 4, 3, 2, 1}}:

\begin{sphinxVerbatim}[commandchars=\\\{\}]
int[] arr = \PYGZob{} 1, 2, 3, 4, 5 \PYGZcb{};
for (???) \PYGZob{}
    System.out.println(arr[i]);
\PYGZcb{}
\end{sphinxVerbatim}
\end{quote}


\subsection{Exchanging and shuffling}
\label{\detokenize{arrays-arraylists:exchanging-and-shuffling}}
Two common tasks when manipulating arrays are \sphinxstyleemphasis{exchanging two values} and \sphinxstyleemphasis{shuffling} values. (\sphinxstyleemphasis{Sorting} is more complicated and will be address later.)

To exchange to values, consider the following code:

\begin{sphinxVerbatim}[commandchars=\\\{\}]
\PYG{n}{double}\PYG{p}{[}\PYG{p}{]} \PYG{n}{arr} \PYG{o}{=} \PYG{p}{\PYGZob{}} \PYG{l+m+mf}{1.0}\PYG{p}{,} \PYG{l+m+mf}{2.0}\PYG{p}{,} \PYG{l+m+mf}{3.0}\PYG{p}{,} \PYG{l+m+mf}{4.0}\PYG{p}{,} \PYG{l+m+mf}{5.0}\PYG{p}{,} \PYG{l+m+mf}{6.0} \PYG{p}{\PYGZcb{}}\PYG{p}{;}
\PYG{n+nb}{int} \PYG{n}{i} \PYG{o}{=} \PYG{l+m+mi}{1}\PYG{p}{;}
\PYG{n+nb}{int} \PYG{n}{j} \PYG{o}{=} \PYG{l+m+mi}{4}\PYG{p}{;}
\PYG{n}{double} \PYG{n}{tmp} \PYG{o}{=} \PYG{n}{arr}\PYG{p}{[}\PYG{n}{i}\PYG{p}{]}\PYG{p}{;}
\PYG{n}{arr}\PYG{p}{[}\PYG{n}{i}\PYG{p}{]} \PYG{o}{=} \PYG{n}{arr}\PYG{p}{[}\PYG{n}{j}\PYG{p}{]}\PYG{p}{;}
\PYG{n}{arr}\PYG{p}{[}\PYG{n}{j}\PYG{p}{]} \PYG{o}{=} \PYG{n}{tmp}\PYG{p}{;}
\end{sphinxVerbatim}
\begin{quote}

\sphinxstylestrong{Exercise:} What are the six values in the array, in order?
\end{quote}

To shuffle the array, consider the following code:

\begin{sphinxVerbatim}[commandchars=\\\{\}]
\PYG{n+nb}{int} \PYG{n}{n} \PYG{o}{=} \PYG{n}{arr}\PYG{o}{.}\PYG{n}{length}\PYG{p}{;}
\PYG{k}{for} \PYG{p}{(}\PYG{n+nb}{int} \PYG{n}{i} \PYG{o}{=} \PYG{l+m+mi}{0}\PYG{p}{;} \PYG{n}{i} \PYG{o}{\PYGZlt{}} \PYG{n}{n}\PYG{p}{;} \PYG{n}{i}\PYG{o}{+}\PYG{o}{+}\PYG{p}{)} \PYG{p}{\PYGZob{}}
    \PYG{n+nb}{int} \PYG{n}{r} \PYG{o}{=} \PYG{n}{i} \PYG{o}{+} \PYG{p}{(}\PYG{n+nb}{int}\PYG{p}{)} \PYG{p}{(}\PYG{n}{Math}\PYG{o}{.}\PYG{n}{random}\PYG{p}{(}\PYG{p}{)} \PYG{o}{*} \PYG{p}{(}\PYG{n}{n}\PYG{o}{\PYGZhy{}}\PYG{n}{i}\PYG{p}{)}\PYG{p}{)}\PYG{p}{;}
    \PYG{n}{String} \PYG{n}{tmp} \PYG{o}{=} \PYG{n}{arr}\PYG{p}{[}\PYG{n}{r}\PYG{p}{]}\PYG{p}{;}
    \PYG{n}{arr}\PYG{p}{[}\PYG{n}{r}\PYG{p}{]} \PYG{o}{=} \PYG{n}{arr}\PYG{p}{[}\PYG{n}{i}\PYG{p}{]}\PYG{p}{;}
    \PYG{n}{arr}\PYG{p}{[}\PYG{n}{i}\PYG{p}{]} \PYG{o}{=} \PYG{n}{tmp}\PYG{p}{;}
\PYG{p}{\PYGZcb{}}
\end{sphinxVerbatim}
\begin{quote}

\sphinxstylestrong{Question:} What does this code do:

\begin{sphinxVerbatim}[commandchars=\\\{\}]
\PYG{k}{for} \PYG{p}{(}\PYG{n+nb}{int} \PYG{n}{i} \PYG{o}{=} \PYG{l+m+mi}{0}\PYG{p}{;} \PYG{n}{i} \PYG{o}{\PYGZlt{}} \PYG{n}{n}\PYG{o}{/}\PYG{l+m+mi}{2}\PYG{p}{;} \PYG{n}{i}\PYG{o}{+}\PYG{o}{+}\PYG{p}{)} \PYG{p}{\PYGZob{}}
    \PYG{n}{double} \PYG{n}{tmp} \PYG{o}{=} \PYG{n}{arr}\PYG{p}{[}\PYG{n}{i}\PYG{p}{]}\PYG{p}{;}
    \PYG{n}{arr}\PYG{p}{[}\PYG{n}{i}\PYG{p}{]} \PYG{o}{=} \PYG{n}{arr}\PYG{p}{[}\PYG{n}{n}\PYG{o}{\PYGZhy{}}\PYG{l+m+mi}{1}\PYG{o}{\PYGZhy{}}\PYG{n}{i}\PYG{p}{]}\PYG{p}{;}
    \PYG{n}{arr}\PYG{p}{[}\PYG{n}{n}\PYG{o}{\PYGZhy{}}\PYG{n}{i}\PYG{o}{\PYGZhy{}}\PYG{l+m+mi}{1}\PYG{p}{]} \PYG{o}{=} \PYG{n}{tmp}\PYG{p}{;}
\PYG{p}{\PYGZcb{}}
\end{sphinxVerbatim}
\end{quote}


\subsection{5. Exercises}
\label{\detokenize{arrays-arraylists:exercises}}\begin{enumerate}
\def\theenumi{\arabic{enumi}}
\def\labelenumi{\theenumi .}
\makeatletter\def\p@enumii{\p@enumi \theenumi .}\makeatother
\item {} 
Write a program that reverses the order of values in an array.

\item {} 
What is wrong with this code snippet?
\begin{quote}

\begin{sphinxVerbatim}[commandchars=\\\{\}]
\PYG{n+nb}{int}\PYG{p}{[}\PYG{p}{]} \PYG{n}{arr}\PYG{p}{;}
\PYG{k}{for} \PYG{p}{(}\PYG{n+nb}{int} \PYG{n}{i} \PYG{o}{=} \PYG{l+m+mi}{0}\PYG{p}{;} \PYG{n}{i} \PYG{o}{\PYGZlt{}} \PYG{l+m+mi}{10}\PYG{p}{;} \PYG{n}{i}\PYG{o}{+}\PYG{o}{+}\PYG{p}{)} \PYG{p}{\PYGZob{}}
    \PYG{n}{arr}\PYG{p}{[}\PYG{n}{i}\PYG{p}{]} \PYG{o}{=} \PYG{n}{i}\PYG{p}{;}
\PYG{p}{\PYGZcb{}}
\end{sphinxVerbatim}
\end{quote}

\item {} 
Write a program \sphinxcode{\sphinxupquote{HowMany.java}} that takes an arbitrary number of command line arguments and prints how many there are.

\end{enumerate}


\subsection{References}
\label{\detokenize{arrays-arraylists:references}}\begin{itemize}
\item {} 
\sphinxhref{https://introcs.cs.princeton.edu/java/14array/}{Computer Science: An Interdisciplinary Approach}, Robert Sedgewick and Kevin Wayne.

\end{itemize}


\chapter{Textbook}
\label{\detokenize{index:textbook}}

\chapter{For students}
\label{\detokenize{index:for-students}}

\chapter{For instructors}
\label{\detokenize{index:for-instructors}}
To build this documentation in HTML, \sphinxhref{http://www.sphinx-doc.org/en/master/usage/restructuredtext/basics.html}{install Sphinx} and run

\begin{sphinxVerbatim}[commandchars=\\\{\}]
\PYG{n}{make} \PYG{n}{html}
\end{sphinxVerbatim}

inside the \sphinxcode{\sphinxupquote{docs}} subdirectory.

To modify styling, edit \sphinxcode{\sphinxupquote{style.css}} in the \sphinxcode{\sphinxupquote{docs/source/\_static}} directory.

To build this documentation as a PDF, run

\begin{sphinxVerbatim}[commandchars=\\\{\}]
\PYG{n}{make} \PYG{n}{latexpdf}
\end{sphinxVerbatim}

inside the \sphinxcode{\sphinxupquote{docs}} subdirectory.



\renewcommand{\indexname}{Index}
\printindex
\end{document}