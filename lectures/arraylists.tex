\chapter{ArrayLists}

A \emph{collection} is a group of objects. Today, we'll be looking at a very useful collection, the \ic{ArrayList}. A \emph{list} is an ordered collection, and an \ic{ArrayList} is one type of list.

Create a class \ic{NameTracker} and follow along in it.

Before we can use an ArrayList, we have to import it:

\begin{code}
import java.util.ArrayList;
\end{code}

Next, we call the constructor; but we have to declare the type of object the :code:`ArrayList` is going to hold. This is how you create a new :code:`ArrayList` holding :code:`String` objects.

\begin{code}
ArrayList<String> names = new ArrayList<String>();
\end{code}

Notice the word ``String'' in angle brackets. This is the Java syntax for constructing an ArrayList of String objects.

We can add a new String to names using the :code:`add()` method.

\begin{code}
names.add("Ana");
\end{code}

\begin{example}
Exercise: Write a program that asks the user for some names and then stores them in an ArrayList. Here is an example program:

\begin{monospace}
Please give me some names:
Sam
Alecia
Trey
Enrique
Dave
Your name(s) are saved!
\end{monospace}
\end{example}

We can see how many objects are in our ArrayList using the size() method.

\begin{code}
System.out.println(names.size()); // 5
\end{code}

\begin{example}
Modify your program to notify the user how many words they have added.

\begin{monospace}
Please give me some names:
Mary
Judah
Your 2 name(s) are saved!
\end{monospace}
\end{example}

Remember how the \ic{String.charAt()} method returns the \ic{char} at a particular index? We can do the same with names. Just call \ic{get()}:

\begin{code}
names.add("Noah");
names.add("Jeremiah");
names.add("Ezekiel");
System.out.println(names.get(2)); // ``Ezekiel''
\end{code}

\begin{example}
Update your program to repeat the names back to the user in reverse order. Your solution should use a for loop and the size() method. For example:

\begin{monospace}
Please give me some names:
Ying
Jordan

Your 2 name(s) are saved! They are:
Jordan
Ying
\end{monospace}
\end{example}

Finally, we can ask our names \ic{ArrayList} whether or not it has a particular string.

\begin{code}
names.add(``Veer'');
System.out.println(names.contains(``Veer'')); // true
\end{code}

\begin{example}
Update your program to check if a name was input by the user. For example:

\begin{monospace}
Please give me some names:
Ying
Jordan

Search for a name:
Ying
Yes!
\end{monospace}
\end{example}

An \ic{ArrayList} can hold any type of object! For example, here is a constructor for an \ic{ArrayList} holding an instance of a \ic{Person} class:

\begin{code}
ArrayList<Person> people = new ArrayList<Person>();
\end{code}

where \ic{Person} is defined as

\begin{code}
public class Person {

    String name;
    int age;

    public Person(String name, int age) {
        this.name = name;
        this.age = age;
    }

    public String getName() {
        return this.name;
    }

    public int getAge() {
        return this.age;
    }
}
\end{code}

\begin{example}
Modify our program to save the user's input names as Person instances. Rather than storing String objects in the ArrayList, store Person objects by constructing them with the input name. You'll need to use the Person constructor to get a Person instance!
\end{example}

\exercisesection

\begin{exercise}

Write a class BlueBook that tells the user the price of their car, depending on the make, model, and year. You should use Car.java and the stencil file provided, BlueBook.java.

Your program depends on what cars your BlueBook supports, but here is an example program:

\begin{monospace}
What is your car's make?
Toyota
What is your Toyota's model?
Corolla
What is your Toyota Corolla's year?
1999

Your 1999 Toyota Corolla is worth \$2000.
\end{monospace}

\end{exercise}

\begin{exercise}
Notify the user if the car is not in your BlueBook.
\end{exercise}

\begin{exercise}
Clean up main by putting your code for creating the ArrayList in a separate method. What type should the method return?
\end{exercise}

\begin{exercise}
If the car is not in the BlueBook, ask the user to input the relevant data, construct a new Car instance, add it to your ArrayList.
\end{exercise}

\referencessection

1. \url{https://github.com/accesscode-2-1/unit-0/blob/master/lessons/week-3/2015-03-24_arraylists.md}